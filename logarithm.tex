\documentclass[11pt,a4paper]{jsarticle}
%
\usepackage{amsmath,amssymb}
\usepackage{bm}
\usepackage{graphicx}
\usepackage{ascmac}
\usepackage{tkySty/slashbox}
\usepackage{tkySty/TkyMemo}
%
\setlength{\textwidth}{\fullwidth}
\setlength{\textheight}{40\baselineskip}
\addtolength{\textheight}{\topskip}
\setlength{\voffset}{-0.2in}
\setlength{\topmargin}{0pt}
\setlength{\headheight}{0pt}
\setlength{\headsep}{0pt}
%
\newcommand{\divergence}{\mathrm{div}\,}  %ダイバージェンス
\newcommand{\grad}{\mathrm{grad}\,}  %グラディエント
\newcommand{\rot}{\mathrm{rot}\,}  %ローテーション
%
\title{指数、対数について from kindle}
\author{...}
\date{\today}
\begin{document}
\maketitle
%
%
\section{とりあえずメモ}
\begin{itemize}
    \item 同じ数を掛ける回数のことを\TkyMemo{指数}{} 
    \item 例えば(10に何乗すれば良いかを表すのが\TkyMemo{対数}{} 
    \item 対数は29乗などのような計算も一瞬で計算できるようにする
    \item 放射性物質の崩壊は1/2の繰り返し、これらから、化石の時代を推定することが可能
    \item 放射性物質は、原子核が\TkyMemo{崩壊}して他の原子核に変化する p28
    \item 元の1/2になる期間のことをを\TkyMemo{半減期}という
\end{itemize}

%
%
\end{document}
\documentclass[11pt,a4paper]{jsarticle}
%
\usepackage{amsmath,amssymb}
\usepackage{bm}
\usepackage{graphicx}
\usepackage{ascmac}
%
\setlength{\textwidth}{\fullwidth}
\setlength{\textheight}{40\baselineskip}
\addtolength{\textheight}{\topskip}
\setlength{\voffset}{-0.2in}
\setlength{\topmargin}{0pt}
\setlength{\headheight}{0pt}
\setlength{\headsep}{0pt}
%
\newcommand{\divergence}{\mathrm{div}\,}  %ダイバージェンス
\newcommand{\grad}{\mathrm{grad}\,}  %グラディエント
\newcommand{\rot}{\mathrm{rot}\,}  %ローテーション
%
\title{...}
\author{...}
\date{\today}
\begin{document}
\maketitle
%
%
\section{...}
...
...

%
%
\end{document}

