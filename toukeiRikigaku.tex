\documentclass[11pt,a4paper]{jsarticle}
%
\usepackage{amsmath,amssymb}
\usepackage{bm}
\usepackage{graphicx}
\usepackage{ascmac}
\usepackage{tkySty/slashbox}
\usepackage{tkySty/TkyMemo}
%
\setlength{\textwidth}{\fullwidth}
\setlength{\textheight}{40\baselineskip}
\addtolength{\textheight}{\topskip}
\setlength{\voffset}{-0.2in}
\setlength{\topmargin}{0pt}
\setlength{\headheight}{0pt}
\setlength{\headsep}{0pt}
%
\newcommand{\divergence}{\mathrm{div}\,}  %ダイバージェンス
\newcommand{\grad}{\mathrm{grad}\,}  %グラディエント
\newcommand{\rot}{\mathrm{rot}\,}  %ローテーション
%
\title{統計力学入門}
\author{...}
\date{\today}
\begin{document}
\maketitle
%
%
\section{とりあえずメモ}
\begin{itemize}
    \item \TkyMemo{統計力学とは}{熱力学の基本関係式を求めること}
    \item \TkyMemo{統計力学とは}{マクロ変数の平衡値のゆらぎを求めることの2つ}
    \item \TkyMemo{平衡値}{平衡状態における平均値}
    \item 統計力学や熱力学は「系のサイズを大きくしてゆくと、この理論の結果にいくらでも近づく、漸近理論である
        %p.2 まで


    \end{itemize}

%
%
\end{document}





\end{document}

