\documentclass[11pt,a4paper]{jsarticle}
%
\usepackage{amsmath,amssymb}
\usepackage{bm}
\usepackage{graphicx}
\usepackage{ascmac}
%
\setlength{\textwidth}{\fullwidth}
\setlength{\textheight}{40\baselineskip}
\addtolength{\textheight}{\topskip}
\setlength{\voffset}{-0.2in}
\setlength{\topmargin}{0pt}
\setlength{\headheight}{0pt}
\setlength{\headsep}{0pt}
%
\newcommand{\divergence}{\mathrm{div}\,}  %ダイバージェンス
\newcommand{\grad}{\mathrm{grad}\,}  %グラディエント
\newcommand{\rot}{\mathrm{rot}\,}  %ローテーション
%
\title{...}
\author{...}
\date{\today}
\begin{document}
\maketitle
%
%
\section{平均報酬を求めるとき0320}
\begin{eqnarray}
    Q_{k+1} =  Q_k + \frac{1}{k+1}(r_{k+1}-Q_k) 
    % \Delta\epsilon_i &\leftarrow& T(\mu_G - r_i (\epsilon_i)) (r_{ib} - r_{ia}) \\
    % \sigma^2_G = \frac{1}{N}\sum_{}^{N}(\mu_G - r_i (\epsilon_i))^2
    Q_{k+1} &=& \frac{1}{k+1}\sum_{i=1}^{k+1}r_i \\
            &=&  \frac{1}{k+1}(r_{k+1}+\sum_{i=1}^{k}r_i) \\
            &=&  \frac{1}{k+1}(r_{k+1}+kQ_k) \\
            &=&  \frac{1}{k+1}(r_{k+1}+kQ_k+Q_k - Q_k) \\
            &=&  \frac{1}{k+1}(r_{k+1}+(k+1)Q_k-Q_k) \\
            &=&  Q_k + \frac{1}{k+1}(r_{k+1}-Q_k) 
\end{eqnarray}

%
%
\end{document}

